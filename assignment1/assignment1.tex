\documentclass[12pt]{article}
\usepackage[top=1in,bottom=1in,left=1in,right=1in]{geometry}
\usepackage{alltt}
\usepackage{array}	
\usepackage{graphicx}
\usepackage{tabularx}
\usepackage{verbatim}
\usepackage{setspace}
\usepackage{listings}
\usepackage{amssymb,amsmath, amsthm}
\usepackage{qtree}
\usepackage{hyperref}
\usepackage{oz}
\usepackage[cc]{titlepic}
\usepackage{fancyvrb}
\usepackage{epstopdf}
\usepackage{soul}
\graphicspath{ {./figures/} }

\newcolumntype{L}{>{\centering\arraybackslash}m{3cm}}

\title{Concordia University\\
Department of Computer Science and Software Engineering\\
\textbf{SOEN 331 - S\\Formal Methods for Software Engineering}\\
\ \\
\textbf{Assignment 1: Fundamentals}}
\author{\textbf{Witnick-Hans Joseph}\\
		\texttt{ID: 29348743}\\
		\textbf{Alexandra Zana}\\
		\texttt{ID: 40131077}\\
		\textbf{Alexandre Eid}\\
		\texttt{ID: 40155833}
\ \\}
\date{October 14, 2021.}

\begin{spacing}{1.5}
\begin{document}
\maketitle

\newpage
\tableofcontents
\newpage

\section{General information}

\noindent \textbf{Date posted}: Thursday 30 September, 2021.\\
\noindent \textbf{Date due}: Thursday, 14 October, 2021, by 23:59.\\
\noindent \textbf{Weight}: 15\% of the overall grade.

\section{Introduction}
You should form a team of \textbf{three} members. Each team should designate a leader who will submit the assignment electronically. In case you cannot find a team, please contact me and I will assign you to one. There are \textbf{7} problems in this assignment, with a total weight of \textbf{100} points. You must prepare all your solutions in~\LaTeX~and produce a single \texttt{pdf} file. Name the file after the Concordia id of the person who will submit, e.g. \texttt{123456.pdf}.\\

\section{Ground rules}

This is an assessment exercise.  You may not seek any assistance while expecting to receive credit. \textbf{You must work strictly within your team and seek no assistance for this assignment ((e.g. from the teaching assistants, fellow classmates and other teams or external help)}. Please note that you should \textbf{not} discuss the assignment during tutorials. I am available to discuss clarifications in case you need any.\\

\noindent \textbf{All team members are expected to work relatively equally on each problem}. The team leader has the responsibility to ensure that the team does not violate this rule. \textbf{In your submission, you must include only the names of those team members who contributed to the assignment.} Accommodating someone who did not contribute will result in a penalty.\\

\noindent If there is any problem in the team (such as lack of contribution, etc.), the team leader must contact the instructor as soon as the problem appears.

\newpage

\section{Problems}

\subsection{Propositional logic (10 pts)}

\noindent You are shown a set of four cards placed on a table, each of which has a \textbf{letter} on one side and a \textbf{symbol} on the other side. The visible faces of the cards show the letters \textbf{L} and \textbf{A}, and the symbols \textbf{$\Box$}, and \textbf{$\diamondsuit$}.\\

\noindent Which card(s) must you turn over in order to test the truth of the proposition that \textit{``If a card has a consonant on one side, then it has the symbol $\diamondsuit$ on the other side''}? Explain your reasoning \underline{in detail} by deciding for \underline{each} card whether it should be turned over and why. In your answers, apply any and all appropriate validating or non-validating patterns where applicable.\\


\subsubsection{Propositional logic answer}

\noindent We have two statements:

\indent S1: The card has a consonant

\indent S2: The symbol is \textbf{$\diamondsuit$}

$ S1 \rightarrow S2$

\noindent Card \textbf{$\diamondsuit$}: Implication does not necessarily entail causation. So even if the card is a \textbf{$\diamondsuit$} the othe side value could be L or A. It is not the card to turn.

\noindent Card \textbf{$\square$}: This card does not satisfy the second statement that the card is a \textbf{$\diamondsuit$}. So we cannot evaluate the second statement. It is not the card to turn.

\noindent Card A: Since this card is a vowel, it will not help in satisfying the proposition of S1. It is not the card to turn.

\noindent Card L: This card will help answer the question because it satisfy the premise that the card has a consonant. We can check the back to validate S2. This should be the card to turn.

\newpage

\subsection{Predicate logic 1 (10 pts)}

In the domain of all people, consider the predicate \textit{disclosed(a, b)} that is interpreted as \textit{``a has disclosed a secret to  b.''}
\begin{enumerate}
\item How are the following two expressions translated into plain English? Are the two expressions logically equivalen $\forall a \exists b~asks(a, b)$, and $\exists b \forall a ~asks(a, b)$.
\begin{itemize}
	\item $\forall a \exists b~disclosed(a, b)$.
	\item $\exists b \forall a ~disclosed(a, b)$.
\end{itemize} 
\item Can we claim that $\forall a \exists b~disclosed(a, b) \rightarrow \exists b \forall a ~disclosed(a, b)$? Discuss in detail.
\item Can we claim that $\exists b \forall a ~disclosed(a, b) \rightarrow \forall a \exists b~disclosed(a, b)$? Discuss in detail.
\end{enumerate}

\subsubsection{Problem 2 answer}
\begin{enumerate}
\item translation
	\subitem $\forall a \exists b~disclosed(a, b)$: For all $a$, there exist a $b$ such that $a$ has asked $b$ out on a date. 
	\subitem $\exists b \forall a ~disclosed(a, b)$: There exists a $b$ such that all $a$ asked $b$ out on a date.
\item Can we claim that $\forall a \exists b~disclosed(a, b) \rightarrow \exists b \forall a ~disclosed(a, b)$? Discuss in detail.
	\subitem  
\item Can we claim that $\exists b \forall a ~disclosed(a, b) \rightarrow \forall a \exists b~disclosed(a, b)$? Discuss in detail.
\end{enumerate} 


\newpage

\subsection{Predicate logic 2 (10 pts)}

\noindent Consider the subject ``x is a person'' and the predicate ``x is a mortal'', together with the following list of categorical propositions:
\begin{itemize}
\item ``No person is immortal.''

\item ``All people is immortal.''

\item ``Some people are mortal.''	

\item ``Some people are not mortal.''	
\end{itemize} 


\begin{enumerate}
\item  ``Identify each categorical statement with its name (i.e. letter description)''.
\subitem answer: 

\item ``Identify universal statements.''
\subitem answer:  

\item ``Identify particular statements.''
\subitem answer:  

\item ``Identify affirmative statements''
\subitem answer: 

\item ``Some scientists are honest.''
\subitem answer:

\item ``Identify negative statements.''
\subitem answer: 

\item ``Identify statements with opposite truth values''
\subitem answer:  

\item ``Identify statements that cannot both be true, but could both be false.''
\subitem answer:  

\item ``Identify statements that cannot both be false but could both be true.''
\subitem answer:  

\item ``Identify pairs of super-subaltern statements.''
\subitem answer:  

\end{enumerate}


\newpage

\subsection{Ordered structures (10 pts)}

Consider a list $\Lambda = \langle w, x, y, z \rangle$, deployed to implement a stack Abstract Data Type.

\begin{enumerate}

\item Let the head of $\Lambda$ correspond to the topmost position of the Stack. Implement the body of operations \texttt{push(el, $\Lambda$)} and \texttt{pop($\Lambda$)} (let return element be held in variable topmost) using list construction operations. In both cases a)  we assume that appropriate preconditions exist, and b) we can refer to $\Lambda'$ as the state of the list upon successful termination of one of its operations.

\noindent answer: 

\indent \texttt{push(el, $\Lambda$)}  can be written as $\Lambda' = cons( el, \Lambda) = \langle el, w, x, y, z \rangle$.


\indent \texttt{pop($\Lambda$)}  can be written as $head(\Lambda) = element = w$ and $\Lambda' = tail(\Lambda) = \langle x, y, z \rangle$,

\item  Let the last element of $\Lambda$ correspond to the topmost position of the Stack. Implement the body of both operations as above. When applicable, use control flow statements in your answer.

\end{enumerate}

\newpage

\subsection{Unordered structures and type declarations (10 pts)}

\newpage

\subsection{Relational calculus 1 (25 pts)}
Consider a system that assigns id’s to locations. An id may represent a vehicle in a parking
lot, a train in a station, a process in an operating system etc. A location may represent a
parking spot, a train station platform, a core in a hardware system etc. The requirements
of the system are as follows:
\begin{enumerate}
	\item Id’s are unique.
	\item An id is assigned to a single location (and maybe re-assigned subsequently).
	\item No multiple id’s may be assigned to the same location.
\end{enumerate}
The model of the system is captured by a relation assignment, which is represented as a set
as shown below:
\[
assignment = \\
\{ \\
\hspace{10mm} 001 \mapsto A,\\
\hspace{10mm} 002 \mapsto B,\\
\hspace{10mm} 003 \mapsto C\\
\}
\]

\begin{enumerate}
	\item Define the precondition to operation add, that assigns a new id to a location.
	\item Provide two alternative definitions for the main functionality of the operation.
	\item What would be the result of calling the operation with id? = 001, and location? = D?
	\item Let us get rid of the precondition. What would be the result of calling add with\\ id? = 004, and location? = C? Would you accept or reject the call? If accepted then the pair should be added to the relation. If rejected, it should not be added to the relation.
	\item Assume that in the absence of a precondition, we attempted to call operation add with some input parameters id? and location?. Under what conditions, if any, would this be acceptable?
	\item Consider operation reassign that modifies the location of an existing id. Define the precondition to operation reassign.
	\item Define the body of operation reassign.
\end{enumerate}
\newpage

\subsection{Relational calculus 2 (25 pts)}
Consider the following relation:

\[ cellphones : Model \leftrightarrow Brand \]

\noindent where

\[
cellphones = \\
\{ \\
\hspace{10mm} galaxyA21s \mapsto samsung,\\
\hspace{10mm} mate10 \mapsto huawei,\\
\hspace{10mm} mate10pro \mapsto huawei,\\
\hspace{10mm} galaxyA01 \mapsto samsung,\\
\hspace{10mm} iPhone12ProMax \mapsto apple,\\
\hspace{10mm} iPhoneSE \mapsto apple,\\
\hspace{10mm} galaxyJ \mapsto samsung\\
\hspace{10mm} redmi6A \mapsto xiaomi\\
\hspace{10mm} redmi8 \mapsto xiaomi\\
\}
\]
\begin{enumerate}
	\item Given the above relation,
	\begin{enumerate}
		\item Define the domain of cellphones.\\
		answer: \\
		\indent domain cellphones = Model = \{galaxyA21s, mate10, mate10pro, galaxyA01, iPhone12ProMax, iPhoneSE 7, galaxyJ 2Core 7, redmi6A, redmi8 7\}		
		\item Define the range of cellphones.\\
		answer: range cellphones = Brand = {samsung, huawei, apple, xiaomi}
	\end{enumerate}

	\item Given the following the expression:\\ \{galaxyA01, galaxyJ 2Core, redmi8\} $\lhd$ cellphones
	\begin{enumerate}
		\item What is the meaning of operator $\lhd$? \\
			answer: $\vartriangleleft$ is a domain restriction, it selects pairs according to the first element
		\item What is the value of the expression?	\\
		\indent answer:
		\indent \{galaxyA01, galaxyJ2Core, redmi8\} $\vartriangleleft$ cellphones = \{
		\newline	 galaxyA01 $\mapsto$ samsung,
		\newline	 galaxyJ2Core $\mapsto$ samsung,
		\newline	 redmi8 $\mapsto$ xiaomi	
		\newline \}
		\item Where would you deploy the operator $\lhd$ in the context of a database management system?\\
		answer: This operator is used in database queries
	\end{enumerate}

	\item Given the following the expression:\\ cellphones $\rhd$ \{samsung, xiaomi\}
	\begin{enumerate} 
		\item What is the meaning of operator $\rhd$? \\
		answer: 
		\item What is the value of the expression?	
		\item Where would you deploy such operator in the context of a database management system?
	\end{enumerate}

	\item Given the following the expression:\\ \{mate10pro, iPhoneSE, galaxyJ 2Core\} $\ndres$ cellphones 
	\begin{enumerate} 
		\item What is the meaning of operator $\ndres$? answer: 
		\item What is the value of the expression?	
		\item Where would you deploy such operator in the context of a database management system?
	\end{enumerate}

	\item Given the following the expression:\\ cellphones $\nrres$ \{apple, xiaomi\}
	\begin{enumerate} 
		\item What is the meaning of operator $\nrres$? answer: 
		\item What is the value of the expression?	
		\item Where would you deploy such operator in the context of a database management system?
	\end{enumerate}

	\item Given the following the expression:\\ cellphones $\oplus$ \{galaxyA51 7→ samsung, mate9 7→ huawei\}
	\begin{enumerate} 
		\item What is the meaning of operator $\oplus$? 
		\indent answer: \\
		\indent $\bigoplus$ is a relational overide

		\item What is the value of the expression?	\\
		\indent cellphones $\bigoplus$ \{galaxyA51 $\mapsto$ samsung, mate9 $\mapsto$ huawei\} = \{
		\newline  	galaxyA21s $\mapsto$ samsung
		\newline	mate10 $\mapsto$ huawei
		\newline	mate10pro $\mapsto$ huawei
		\newline	galaxyA01 $\mapsto$ samsung
		\newline	iphone12ProMax $\mapsto$ apple
		\newline	iphoneSE $\mapsto$ apple
		\newline	galaxyJ2Core $\mapsto$ samsung
		\newline	redmi6A $\mapsto$ xiaomi
		\newline	redmi8 $\mapsto$ xiaomi
		\newline	galaxyA51 $\mapsto$ samsung
		\newline	mate9 $\mapsto$ huawei
		\newline \}

		\item Where would you deploy such operator in the context of a database management system?\\
		\indent answer: \\
		\indent This operator would be used in the event that we want to add or modify elements in the database.

		\item  Does the result of the expression have a permanent effect on the database (relation)? If not, describe in detail how would you ensure that such operation would have a permanent effect.\\
		\indent answer:\\
		\indent The expression doesn't have a permanent effect. If we want to have a permanent effect, we would need to have an assignment operation as such:\\
		\indent cellphones' = cellphones $\bigoplus$ \{galaxyA51 $\mapsto$ samsung, mate9 $\mapsto$ huawei\}\\
		\indent We need to assign the value of the expression to cellphone to have permanent effect.\\
		
	\end{enumerate}
\end{enumerate}

\newpage

\section{What to submit}

\noindent Please submit your \texttt{pdf} file at the Electronic Assignment Submission portal

\centerline{(\url{https://fis.encs.concordia.ca/eas}) }

\noindent under \textbf{Theory Assignment 1}.

\end{spacing}

\end{document}