\documentclass[12pt]{article}
\usepackage[top=1in,bottom=1in,left=1in,right=1in]{geometry}
\usepackage{alltt}
\usepackage{array}	
\usepackage{graphicx}
\usepackage{tabularx}
\usepackage{verbatim}
\usepackage{setspace}
\usepackage{listings}
\usepackage{amssymb,amsmath, amsthm}
\usepackage{qtree}
\usepackage{hyperref}
\usepackage{oz}
\usepackage[cc]{titlepic}
\usepackage{fancyvrb}
\usepackage{epstopdf}
\usepackage{soul}
\graphicspath{ {./figures/} }

\newcolumntype{L}{>{\centering\arraybackslash}m{3cm}}

\title{Concordia University\\
Department of Computer Science and Software Engineering\\
\textbf{SOEN 331 - S\\Formal Methods for Software Engineering}\\
\ \\
\textbf{Assignment 1: Fundamentals}}
\author{\textbf{Witnick-Hans Joseph}\\
		\texttt{ID: 29348743}\\
		\textbf{Alexandra Zana}\\
		\texttt{ID: 123}\\
		\textbf{Alexandre Eid}\\
		\texttt{ID: 456}
\ \\}
\date{October 14, 2021.}

\begin{spacing}{1.5}
\begin{document}
\maketitle

\newpage
\tableofcontents
\newpage

\section{General information}

\noindent \textbf{Date posted}: Thursday 30 September, 2021.\\
\noindent \textbf{Date due}: Thursday, 14 October, 2021, by 23:59.\\
\noindent \textbf{Weight}: 15\% of the overall grade.

\section{Introduction}
You should form a team of \textbf{three} members. Each team should designate a leader who will submit the assignment electronically. In case you cannot find a team, please contact me and I will assign you to one. There are \textbf{7} problems in this assignment, with a total weight of \textbf{100} points. You must prepare all your solutions in~\LaTeX~and produce a single \texttt{pdf} file. Name the file after the Concordia id of the person who will submit, e.g. \texttt{123456.pdf}.\\

\section{Ground rules}

This is an assessment exercise.  You may not seek any assistance while expecting to receive credit. \textbf{You must work strictly within your team and seek no assistance for this assignment ((e.g. from the teaching assistants, fellow classmates and other teams or external help)}. Please note that you should \textbf{not} discuss the assignment during tutorials. I am available to discuss clarifications in case you need any.\\

\noindent \textbf{All team members are expected to work relatively equally on each problem}. The team leader has the responsibility to ensure that the team does not violate this rule. \textbf{In your submission, you must include only the names of those team members who contributed to the assignment.} Accommodating someone who did not contribute will result in a penalty.\\

\noindent If there is any problem in the team (such as lack of contribution, etc.), the team leader must contact the instructor as soon as the problem appears.

\newpage

\section{Problems}

\subsection{Propositional logic (10 pts)}

\noindent You are shown a set of four cards placed on a table, each of which has a \textbf{letter} on one side and a \textbf{symbol} on the other side. The visible faces of the cards show the letters \textbf{L} and \textbf{A}, and the symbols \textbf{$\Box$}, and \textbf{$\diamondsuit$}.\\

\noindent Which card(s) must you turn over in order to test the truth of the proposition that \textit{``If a card has a consonant on one side, then it has the symbol $\diamondsuit$ on the other side''}? Explain your reasoning \underline{in detail} by deciding for \underline{each} card whether it should be turned over and why. In your answers, apply any and all appropriate validating or non-validating patterns where applicable.\\


\subsubsection{Propositional logic answer}

\noindent We have two statements:

\indent S1: The card has a consonant

\indent S2: The symbol is \textbf{$\diamondsuit$}

$ S1 \rightarrow S2$

\noindent Card \textbf{$\diamondsuit$}: Implication does not necessarily entail causation. So even if the card is a \textbf{$\diamondsuit$} the othe side value could be L or A. It is not the card to turn.

\noindent Card \textbf{$\square$}: This card does not satisfy the second statement that the card is a \textbf{$\diamondsuit$}. So we cannot evaluate the second statement. It is not the card to turn.

\noindent Card A: Since this card is a vowel, it will not help in satisfying the proposition of S1. It is not the card to turn.

\noindent Card L: This card will help answer the question because it satisfy the premise that the card has a consonant. We can check the back to validate S2. This should be the card to turn.

\newpage

\subsection{Predicate logic 1 (10 pts)}

In the domain of all people, consider the predicate \textit{disclosed(a, b)} that is interpreted as \textit{``a has disclosed a secret to  b.''}
\begin{enumerate}
\item How are the following two expressions translated into plain English? Are the two expressions logically equivalen $\forall a \exists b~asks(a, b)$, and $\exists b \forall a ~asks(a, b)$.
\begin{itemize}
	\item $\forall a \exists b~disclosed(a, b)$.
	\item $\exists b \forall a ~disclosed(a, b)$.
\end{itemize} 
\item Can we claim that $\forall a \exists b~disclosed(a, b) \rightarrow \exists b \forall a ~disclosed(a, b)$? Discuss in detail.
\item Can we claim that $\exists b \forall a ~disclosed(a, b) \rightarrow \forall a \exists b~disclosed(a, b)$? Discuss in detail.
\end{enumerate}

\subsubsection{Problem 2 answer}
\begin{enumerate}
\item translation
	\subitem $\forall a \exists b~disclosed(a, b)$: For all $a$, there exist a $b$ such that $a$ has asked $b$ out on a date. 
	\subitem $\exists b \forall a ~disclosed(a, b)$: There exists a $b$ such that all $a$ asked $b$ out on a date.
\item Can we claim that $\forall a \exists b~disclosed(a, b) \rightarrow \exists b \forall a ~disclosed(a, b)$? Discuss in detail.
	\subitem  
\item Can we claim that $\exists b \forall a ~disclosed(a, b) \rightarrow \forall a \exists b~disclosed(a, b)$? Discuss in detail.
\end{enumerate} 


\newpage

\subsection{Predicate logic 2 (10 pts)}

\noindent Consider the subject ``x is a person'' and the predicate ``x is a mortal'', together with the following list of categorical propositions:
\begin{itemize}
\item ``No person is immortal.''

\item ``All people is immortal.''

\item ``Some people are mortal.''	

\item ``Some people are not mortal.''	
\end{itemize} 


\begin{enumerate}
\item  ``Identify each categorical statement with its name (i.e. letter description)''.
\subitem answer: 

\item ``Identify universal statements.''
\subitem answer:  

\item ``Identify particular statements.''
\subitem answer:  

\item ``Identify affirmative statements''
\subitem answer: 

\item ``Some scientists are honest.''
\subitem answer:

\item ``Identify negative statements.''
\subitem answer: 

\item ``Identify statements with opposite truth values''
\subitem answer:  

\item ``Identify statements that cannot both be true, but could both be false.''
\subitem answer:  

\item ``Identify statements that cannot both be false but could both be true.''
\subitem answer:  

\item ``Identify pairs of super-subaltern statements.''
\subitem answer:  

\end{enumerate}

\noindent Identify pairs that are contradictories, contraries, subcontraries, and pairs that support subalteration (clearly indicating superaltern and subaltern).\\
\indent answer: 

\indent contradictories: (1, 5), (2 ,5), (3, 8), (6, 8)

\indent contraries: (1, 3), (2, 3), (1, 6), (2, 6)

\indent subcontraries: (4, 5), (4, 7), (5, 8), (7, 8)

\indent subalteration - superaltern: (1, 4), (2, 4), (3, 5), (3, 7), (1, 8), (2, 8)

\indent subalteration - subaltern: (4, 1), (4, 2), (5, 3), (7, 3), (8, 1), (8, 2)


\newpage

\subsection{Ordered structures (10 pts)}

Consider a list $\Lambda = \langle w, x, y, z \rangle$, deployed to implement a stack Abstract Data Type.

\begin{enumerate}

\item Let the head of $\Lambda$ correspond to the topmost position of the Stack. Implement the body of operations \texttt{push(el, $\Lambda$)} and \texttt{pop($\Lambda$)} (let return element be held in variable topmost) using list construction operations. In both cases a)  we assume that appropriate preconditions exist, and b) we can refer to $\Lambda'$ as the state of the list upon successful termination of one of its operations.

\noindent answer: 

\indent \texttt{push(el, $\Lambda$)}  can be written as $\Lambda' = concat( list(el), list(\Lambda)) = \langle el, w, x, y, z \rangle$.


\indent \texttt{pop($\Lambda$)}  can be written as $head(\Lambda) = element = w$ and $\Lambda' = tail(\Lambda) = \langle x, y, z \rangle$,

\item  Let the last element of $\Lambda$ correspond to the topmost position of the Stack. Implement the body of both operations as above. When applicable, use control flow statements in your answer.

\end{enumerate}

\newpage

\subsection{Unordered structures and type declarations (10 pts)}

\newpage

\subsection{Relational calculus 1 (25 pts)}

\newpage

\subsection{Relational calculus 2 (25 pts)}

\newpage

\section{What to submit}

\noindent Please submit your \texttt{pdf} file at the Electronic Assignment Submission portal

\centerline{(\url{https://fis.encs.concordia.ca/eas}) }

\noindent under \textbf{Theory Assignment 1}.

\end{spacing}

\end{document}